% ------------------------------------------------------------------------
% AMS-LaTeX definitions:     Thesis ++++ Template ************
% ------------------------------------------------------------------------
% Honours Thesis Template Defended on February 2010
% ------------------------------------------------------------------------


\documentclass[12pt]{report}

%=================================================================
% for comments
\newcount\DraftStatus  % 0 suppresses notes to selves in text
% TODO: set to 0 for final version
\DraftStatus=1   
%=================================================================
\usepackage{color}
\definecolor{darkgreen}{rgb}{0,0.5,0}
\definecolor{purple}{rgb}{1,0,1}
% \draftnote{color}{comment} inserts a colored comment in the text
\newcommand{\draftnote}[2]{\ifnum\DraftStatus=1
	\marginpar{
		\tiny\raggedright
		\hbadness=10000
        \def\baselinestretch{0.5}
        \textcolor{#1}{\textsf{\hspace{0pt}#2}}}
     \fi}
% TODO: add yourself here for paper comments registration:
\newcommand{\gangli}[1]{\draftnote{red}{[GLi: #1]}}
\newcommand{\YOURNAME}[1]{\draftnote{blue}{[XX: #1]}}
%other colors include blue, red, purple, cyan, darkgreen, etc.
%=================================================================
%
% for Version control information
% 
\usepackage{svnkw}
%
\svnidlong
{$HeadURL: http://xp-dev.com/svn/TULIP/papers/thesistemplatex/trunk/thesis.tex $}
{$LastChangedDate: 2013-12-16 14:25:19 +1100 (Mon, 16 Dec 2013) $}
{$LastChangedRevision: 277 $}
{$LastChangedBy: gangli $}
\svnid{$Id: thesis.tex 277 2013-12-16 03:25:19Z gangli $}
% TODO: add yourself here for SVN information registration:
\svnRegisterAuthor{gangli}{Gang Li}
\svnRegisterAuthor{ligang}{Gang Li}
\svnRegisterAuthor{YOURACCOUNT}{Your Name}
%\newcommand{\svnfooter}{Last Changed Rev: \svnkw{LastChangedRevision}}
% Don't forget to set the svn property 'svn:keywords' to
% 'HeadURL LastChangedDate LastChangedRevision LastChangedBy' or
% 'Id' or both depending if you use \svnidlong and/or \svnid
% TODO: Include this into Draft Output
%\fancyfoot[LO,LE]{R\svnrev\ (\svnfiledate)}
%\fancyfoot[RO,RE]{Last changed by: \svnFullAuthor{\svnauthor}}



%=================================================================

% ------------------------------------------------------------------------
%=================================================================
% general packages
\usepackage{graphicx,color}
\usepackage{graphics}
\usepackage{algorithm}
\usepackage{algorithmic}
\usepackage[centertags]{amsmath}
\usepackage{amsfonts}
%\usepackage{epic,multibox,fancybox}
\usepackage{lscape}
\usepackage{amssymb}
\usepackage{amsthm}
\usepackage{newlfont}
\usepackage{breqn}
\usepackage{multirow}
\usepackage{subfig}
\usepackage{cite}
\usepackage{url}


\usepackage{xthesis} %DAL Thesis Style
\usepackage{xtocinc} %Include Table of Contents as the first entry in TOC
%                     Faculty of Grad Studies insists on this!?
%\usepackage[active]{srcltx}  %SRC Specials for DVI search
% Fuzz -------------------------------------------------------------------
\hfuzz2pt % Don't bother to report over-full boxes if over-edge is < 2pt
% Line spacing -----------------------------------------------------------
\newlength{\defbaselineskip}
\setlength{\defbaselineskip}{\baselineskip}
\newcommand{\setlinespacing}[1]%
           {\setlength{\baselineskip}{#1 \defbaselineskip}}
\newcommand{\doublespacing}{\setlength{\baselineskip}%
                           {2.0 \defbaselineskip}}
\newcommand{\singlespacing}{\setlength{\baselineskip}{\defbaselineskip}}
%=================================================================
% for math notations
% ----------------------------------------------------------------
\vfuzz2pt % Don't report over-full v-boxes if over-edge is small
\hfuzz2pt % Don't report over-full h-boxes if over-edge is small
% THEOREMS -------------------------------------------------------
\newtheorem{thm}{Theorem}[section]
\newtheorem{cor}[thm]{Corollary}
\newtheorem{lem}[thm]{Lemma}
\newtheorem{prop}[thm]{Proposition}
\theoremstyle{definition}
\newtheorem{defn}[thm]{Definition}
\theoremstyle{remark}
\newtheorem{rem}[thm]{Remark}
\numberwithin{equation}{section}
% MATH -----------------------------------------------------------
\newcommand{\norm}[1]{\left\Vert#1\right\Vert}
\newcommand{\abs}[1]{\left\vert#1\right\vert}
\newcommand{\set}[1]{\left\{#1\right\}}
\newcommand{\Real}{\mathbb R}
\newcommand{\eps}{\varepsilon}
\newcommand{\To}{\longrightarrow}
\newcommand{\BX}{\mathbf{B}(X)}
% ----------------------------------------------------------------
\newcommand{\I}{{\mathcal I}}
\newcommand{\Id}{{\mathcal I} }
\newcommand{\Dc}{{\mathcal D}}
\newcommand{\J}{{\mathcal J}}
\newcommand{\Dn}{{\mathcal D}_n}
\newcommand{\Dd}{{\mathcal D}_n }
\renewcommand{\P}{{\mathcal P}}
\newcommand{\Nu}{{\mathcal N} }
\newcommand{\B}{{\mathcal B}}
\newcommand{\Bf}{{\bf B}}
\newcommand{\Y}{{\bf Y}}
\newcommand{\A}{{\mathcal A}}

\newcommand{\V}{{\mathcal V}}
\newcommand{\M}{{\mathcal M}}
\newcommand{\F}{{\mathcal F}}
\newcommand{\Fd}{{\mathcal F}}
\newcommand{\BF}{{\mathcal BF}_n}
\newcommand{\BFd}{{\mathcal BF}_n}
\newcommand{\TF}{{\mathcal TF}_n}
\newcommand{\TFd}{{\mathcal TF}_n}
\newcommand{\G}{{\mathcal G}}
\newcommand{\X}{{\mathcal X}}
\newcommand{\E}{{\mathcal E}}
\newcommand{\K}{{\mathcal K}}
\newcommand{\T}{{\mathcal T}_n}
\renewcommand{\H}{{\mathcal H}}

\newtheorem{Remark}{Remark}
\newtheorem{proposition}{Proposition}
\newtheorem{theorem}{Theorem}
%\renewcommand{\thetheorem}{\arabic{theorem}}
\newtheorem{lemma}{Lemma}
\newtheorem{corollary}{Corollary}
%\renewcommand{\thelemma}{\arabic{lemma}}
%\renewcommand{\thecorollary}{\arabic{corollary}}
\newtheorem{example}{Example}
%\renewcommand{\theexample}{\arabic{example}}
\newtheorem{definition}{Definition}
%\renewcommand{\thedefinition}{\arabic{definition}}
\newtheorem{Algorithms}{Algorithm}

\newcommand{\bu}{{\mathbf 1} }
\newcommand{\bo}{{\mathbf 0} }
\newcommand{\N}{\mbox{{\sl l}}\!\mbox{{\sl N}}}


\newcommand{\rb}[1]{\raisebox{1.5ex}[0pt]{#1}}

\def\uint{[0,1]}
\def\proof{{\scshape Proof}. \ignorespaces}
\def\endproof{{\hfill \vbox{\hrule\hbox{%
   \vrule height1.3ex\hskip1.0ex\vrule}\hrule
  }}\par}
%=================================================================

\hypersetup
{
    pdfauthor={Gang Li},
    pdfsubject={TULIP Thesis},
    pdftitle={FLIP},
    pdfkeywords={TULIP, Lab}
}


%%% ----------------------------------------------------------------------
\setlength{\tclineskip}{1.05\baselineskip}
%%% ----------------------------------------------------------------------
\begin{document}

%\nobib
%\draft
%\nofront

%\permissionfalse

\dedicate{To \ldots} % dedicate page, for example "To my parents, To my family, ......"

%\nolistoftables

%\nolistoffigures

\copyrightyear{201X} 
\submitdate{Month 201X}
\convocation{Month}{201X}

% \hon  
\phd

% ------------------------------------------------------------------------

\title{Title of your Thesis}

\author{Your full name}

%\twosupervisors

\supervisor{Your principle supervisor's full name}

%\firstreader{Your associate supervisor's full name}



% ------------------------------------------------------------------------
{
\typeout{:?0000} % Don't bother with over/under-full boxes
\beforepreface
\typeout{:?1111} % Process All Errors from Here on
}
% ------------------------------------------------------------------------
%\setcounter{page}{1}
%\tableofcontents

% ------------------------------------------------------------------------
{ \typeout{Acknowledgement}
% Thesis Acknowledgements ----------------------------------------------

\prefacesection{Acknowledgements}
\def\baselinestretch{1.66}
\setlinespacing{1.15}

I would like to thank Dr. Gang Li, my supervisor, 
for his many suggestions and constant support during this research. \\

I am also thankful to XXX for useful suggestions and friendly encouragement. \\

Of course, 
I am grateful to my parents for their patience and love. 
Without them this work would never have come into existence.

\ldots 

\gangli{You can put your marginal comments like this.}

\ldots 


\bigskip\medskip

\noindent
Melbourne, Australia \hfill 
Your name \\
Date

}

{ \typeout{Publication List}
% Thesis Abstract ------------------------------------------------------

\prefacesection{Publication List}

% Add all your published/accepted papers here, even the ones still in press

\begin{enumerate}
\item paper 1
\item paper 2
\item ......

\end{enumerate}

}

{ \typeout{Abstract}
% Thesis Abstract ------------------------------------------------------

\prefacesection{Abstract}

Add your abstract here ......


\textbf{Keywords:} Machine Learning, Data Mining, ......

}

% ------------------------------------------------------------------------
\afterpreface
\def\baselinestretch{1}
\setlinespacing{1.66}
\cleardoublepage
% ------------------------------------------------------------------------
\def\baselinestretch{1}

\chapter{Introduction}

\def\baselinestretch{1.66}


%%% ----------------------------------------------------------------------

\section{Background and Motivation}




\section{Aims and Research Questions}




\section{Thesis Outline}






















%%% ----------------------------------------------------------------------
 %\cleardoublepage
\def\baselinestretch{1}

\chapter{Literature Review}

\def\baselinestretch{1.66}


%%% ----------------------------------------------------------------------



\section{Introduction}

% Each chapter is started with an introduction section.


%%% ----------------------------------------------------------------------

% add subsections based on your work




%%% ---------------------------------------------------------------------- 

\section{Summary}

% Don't forget to have a brief summary of this chapter and one sentence linked to the next chapter %\cleardoublepage
\def\baselinestretch{1}

\chapter{Your Main Thesis Body $01$}

\def\baselinestretch{1.66}

% This chapter will introduce the experiments based on the proposed methods discussed in Chapter $3$. 
% After the explanation of your method, 
% four major issues will be covered in the following sections: 
% $1$) the experimental data sets; $2$) data pre-processing; 
% $3$) the process of the experiment; and, $4$) the results.


%%% ----------------------------------------------------------------------

\section{Introduction}

% Each chapter is started with an introduction section.

Start your introduction.


This is the end of overview. Let us start ....

%%% ---------------------------------------------------------------------

\section{Your Method}

% Description of the algorithm, theoretic analysis, etc.



%%% ---------------------------------------------------------------------

\section{Data Sets}

% Description of the raw data set, including data collection if you have this in your work.



%%% ------------------------------------------------------------------------------------
\section{Data Pre-processing}

% The process of the data pre-processing

% Description of the cleared data set, data distribution and structure, attribute list, ...etc.


%%% ----------------------------------------------------------------------
\section{Experimental Design}

% Experiment setting and process, extra sub-datasets

\gangli{You can also put comments on the page margins.}



%%% -----------------------------------------------------------------------

\section{Result Analysis and Discussion}


\subsection{Results}

% Present the experiment results in tables or figures, represent them in English


\subsection{Discussion}

% Discuss the significant findings together with the comparison of the other related methods

% Advantage of your method based on the results obtained

%%% ----------------------------------------------------------------------

\section{Summary}

% Don't forget to have a brief summary of this chapter and one sentence linked to the next chapter 

 %\cleardoublepage
%\include{chapterXX} %\cleardoublepage
\def\baselinestretch{1}

\chapter{Conclusion and Future Work}

\def\baselinestretch{1.66}


%%% ----------------------------------------------------------------------


% the conclusion serves the important function of tying together
% the whole thesis.
%
% In summary form, the developments of the previous chapters should
% be restated, important findings discussed, and conclusions drawn
% from the whole study.
%
% In addition, you may list unanswered questions that have occurred
% during the study, which require further research beyond the
% limits of the project being reported.
%
% The conclusion should leave the reader with the impression of
% completeness and of positive gain.


%%% ----------------------------------------------------------------------
% Summarize the main points of the thesis (arguments)


 %\cleardoublepage

% ------------------------------------------------------------------------
%GATHER{xBib.bib}   % For Gather Purpose Only
%GATHER{Thesis.bbl} % For Gather Purpose Only
\setlinespacing{1.44}
\bibliographystyle{abbrv}
\bibliography{bib/reference}

% ------------------------------------------------------------------------


\end{document}
% ------------------------------------------------------------------------
